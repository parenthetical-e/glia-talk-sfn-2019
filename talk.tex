\documentclass[10pt]{beamer}
\usepackage{appendixnumberbeamer}
\usepackage{booktabs}
\usepackage[scale=2]{ccicons}
\usepackage{pgfplots}
\usepackage{xspace}
\usepackage{xcolor}
\usepackage{caption}

\usetheme[progressbar=frametitle]{metropolis}
\usepgfplotslibrary{dateplot}
\newcommand{\themename}{\textbf{\textsc{metropolis}}\xspace}
\captionsetup[figure]{labelformat=empty}

% ---------------------------------------------------------
\title{Artificial astrocyte networks}
% \subtitle{A modern beamer theme}
% \date{\today}
\date{}
\author{Erik J Peterson}
\institute{CoAxLab\\Carnegie Mellon University}

% ---------------------------------------------------------
\begin{document}
\maketitle

% \begin{frame}{Table of contents.}
%   \setbeamertemplate{section in toc}[sections numbered]
%   \tableofcontents%[hideallsubsections]
% \end{frame}

% ---------------------------------------------------------
\begin{frame}[fragile]{Astrocytes and cognition!?}
\begin{itemize}
    \item Classically astrocytes have been considered neural support cells.
    \item Growing evidence that astrocytes can directly drive cognition and motor behavior.
    \item \alert{Distributed $Ca^{2+}$ dynamics can do computation?}
\end{itemize}
\end{frame}

\begin{frame}[fragile]{Astrocytes and cognition.}
\begin{figure}
    \centering
    \includegraphics[scale=0.45]{images/mu.png} 
    \caption{Neural (\textit{top}) and astrocyte $[Ca^{2+}]$ dynamics (bottom) during frustrated movement. (Mu et al, \textit{Cell} 2019; Taken from Video S2.)}
\end{figure}
\end{frame}

\begin{frame}[fragile]{A theory for astrocytes?}
\begin{figure}
    \centering
    \includegraphics[scale=0.2]{images/agn.png} 
    \caption{
    \begin{itemize}
        \item Very little theoretical study of astrocyte computation
        \item Focus is on neuron-glia interactions.
    \end{itemize}}
\end{figure}
\end{frame}

\begin{frame}[fragile]{Astrocytes and cognition?}
\begin{figure}
    \centering
    \includegraphics[scale=0.45]{images/mu.png} 
    \caption{How to explain this result? We need a new theoretical account?}
\end{figure}
\end{frame}

% ---------------------------------------------------------
\begin{frame}[fragile]{Goal.}
\begin{itemize}
    \item \alert{Develop a new theory for distributed astrocyte computation.}
\end{itemize}
\end{frame}

\begin{frame}[fragile]{Goal.}
\begin{itemize}
    \item So let's do a thought experiment. 
    \item Using a simple model. 
    \item ...Based on artificial neural networks (ANNs)?!
    % \item Use models and proof methods from artificial neural networks to try and set an \alert{upper bound} on astrocyte computation.
\end{itemize}
\end{frame}

% ---------------------------------------------------------
\section[ANNs]{What are artificial neural networks?}
\begin{frame}[fragile]{What are ANNs?}
\begin{itemize}
\item By example, using vision.
\item Sparse networks.
\end{itemize}
\end{frame}

\begin{frame}[fragile]{Visual digit recognition.}
\begin{columns}
\column{0.5\textwidth}
\centering
\includegraphics[scale=0.25]{images/minst.png}
\column{0.5\textwidth}
\centering
 = 1,2,3,4,5,6,\ldots
\end{columns}
\end{frame}

\begin{frame}[fragile]{What are ANNs?}
\begin{columns}
\column{0.5\textwidth}
\begin{figure}
    \centering
    \includegraphics[scale=0.5]{images/nine_only.png} 
\end{figure}
\column{0.5\textwidth}
\centering
 $= 9$?
\end{columns}
\end{frame}


\begin{frame}[fragile]{What are ANNs?}
\begin{columns}
\column{0.5\textwidth}
\centering
\includegraphics[scale=0.5]{images/nine_x.png} 
\column{0.5\textwidth}
\centering
 $\alert{\sum} x_{ij} = 9$
\end{columns}
\end{frame}

\begin{frame}[fragile]{What are ANNs?}
\begin{columns}
\column{0.5\textwidth}
\centering
\includegraphics[scale=0.5]{images/nine_wx.png} 
\column{0.5\textwidth}
\centering
 $\sum \alert{w_{ij}} x_{ij} = 9$
\end{columns}
\end{frame}

\begin{frame}[fragile]{What are ANNs?}
\begin{columns}
\column{0.5\textwidth}
\centering
% \includegraphics[scale=0.5]{images/nine_wx.png} 
\column{0.5\textwidth}
\centering
 $\sum w_{ij} x_{ij} + \alert{b_i} = 9$
\end{columns}
\end{frame}

\begin{frame}[fragile]{What are ANNs?}
\begin{columns}
\column{0.5\textwidth}
\centering
% \includegraphics[scale=0.5]{images/nine_wx.png} 
\column{0.5\textwidth}
\centering
 $\alert{\phi}(\sum w_{ij} x_{ij} + b_i) = 9$
\end{columns}
\end{frame}

\begin{frame}[fragile]{What are ANNs?}
\begin{figure}
    \centering
    \includegraphics[scale=0.4]{images/phi.png}
    \caption{\textit{FI}-curve (left) $\approx$ $\alert{\phi}$ nonlinearity (right)}
\end{figure}
\end{frame}

\begin{frame}[fragile]{What are ANNs?}
\begin{figure}
    \centering
    \includegraphics[scale=0.4]{images/deep.png}
    \caption{
    \alert{A deep ANN network}. Each box is a \textit{layer}: $\phi (\sum w_{ij} x_{ij} + b_i)$
    }
\end{figure}
\end{frame}

% ---------------------------------------------------------
\section[AANs]{Defining artificial astrocyte networks.}

\begin{frame}[fragile]{Basic astrocyte limits.}
\begin{itemize}
    \item No synapses
    \item No axons 
    \item No spikes
\end{itemize}
\end{frame}

\begin{frame}[fragile]{Basic astrocyte limits.}
\begin{itemize}
    \item No spatial $\sum$
    \item No $w_i$
\end{itemize}
\end{frame}

\begin{frame}[fragile]{Basic astrocyte properties}
\begin{itemize}
    \item $[Ca^{2+}]$ dynamics
    \item $[Ca^{2+}]$ dependent gliotransmission
\end{itemize}
\end{frame}

\begin{frame}[fragile]{Recovering \{$\phi, w_i, \sum $\}?}
\begin{itemize}
    \item Let's make three assumptions that let us form a functional analogy between neurons and astrocytes.
\end{itemize}
\end{frame}

\begin{frame}[fragile]{Assumption 1 ($\phi$).}
\begin{figure}
    \centering
    \includegraphics[scale=0.4]{images/phi_ca.png}
    \caption{Firing rate (left) $\leftrightarrow$ $[Ca^{2+}]$ dynamics (right)}
\end{figure}
\begin{itemize}
    \item 
\end{itemize}
\end{frame}

\begin{frame}[fragile]{Assumption 2 ($w_i$).}
\begin{columns}
\column{0.5\textwidth}
\begin{itemize}
    \item \alert{Directional} $[Ca^{2+}]$-dependent gliotransmission.
\end{itemize}
\column{0.5\textwidth}
    \centering
    \includegraphics[scale=0.4]{images/gliotrans.jpeg}
\end{columns}
\end{frame}

\begin{frame}[fragile]{Assumption 3 ($\sum$).}
\begin{itemize}
    \item \alert{Directional} $[Ca^{2+}]$ waves, driven by gliotransmission.
    \begin{itemize}
    \item Axons send spikes across a spatial distances with fidelity.
    \item A $[Ca^{2+}]$ wave can travel a spatial distance with fidelity.
    \end{itemize}
\end{itemize}
\end{frame}

% ---------------------------------------------------------
\section[In theory.]{Theory.}
\begin{frame}[fragile]{Computational calcium waves.}
\begin{itemize}
\item Let's treat neurons only as a source of input; glia do all the computation!
\item Let's study a forward moving $[Ca^{2+}]$ wave.
\item With Assumptions 1-3\ldots
\item \alert{Prove}: our simple model of $[Ca^{2+}]$ waves is a universal function approximator.
\end{itemize}
\end{frame}

\begin{frame}[fragile]{A universal function approximator?}
$\alert{D} - f(x)$ 
\begin{itemize}
    \item[$\alert{D}$] : visual recognition system in fly (target)
\end{itemize}
\end{frame}

\begin{frame}[fragile]{A universal function approximator?}
$D - \alert{f}(x)$ 
\begin{itemize}
\item[$D$] : visual recognition system in fly (\textit{target})\\
\item[$\alert{f}(x)$] : an ANN (\textit{approximator})
% \item[] : $\phi (\sum w_{ij} x_{ij} + b_i)$
\end{itemize}
\end{frame}

\begin{frame}[fragile]{A universal function approximator.}
$|D - f(x)| < \alert{\epsilon}$ 
\begin{itemize}
\item[$D$] : visual recognition system in fly (\textit{target})\\
\item[$f(x)$] : an ANN (\textit{approximator})
\item[$\alert{\epsilon}$] : the max error (a positive number)
\end{itemize}
\end{frame}


\begin{frame}[fragile]{Proof sketch.}
\begin{columns}
\column{0.5\textwidth}
\centering
\begin{figure}
    \centering
    \includegraphics[scale=0.2]{images/ann.png}
    \caption{
    \begin{itemize}
        \item $D$ can be approximated by $f$ in $M << N$ connections.
        \item Bölcskei et al, \textit{J. Math. of Data Science} 2019.
        \item Accept: $|D - f(x)| < \epsilon$ 
    \end{itemize}}
\end{figure}
\column{0.5\textwidth}
\end{columns}
\end{frame}

\begin{frame}[fragile]{Proof sketch.}
\begin{columns}
\column{0.5\textwidth}
\centering
\begin{figure}
    \centering
    \includegraphics[scale=0.2]{images/ann.png}
    \caption{
    \begin{itemize}
        \item $D$ can be approximated by $f$ in $M << N$ connections.
        \item Bölcskei et al, \textit{J. Math. of Data Science} 2019.
        \item Accept: $|D - f(x)| < \epsilon$ 
    \end{itemize}}
\end{figure}
\column{0.5\textwidth}
\centering
\begin{figure}
    \centering
    \includegraphics[scale=0.2]{images/aan.png} 
    \caption{
    \begin{itemize}
        \item If summed ``weight'' of a $Ca^{2+}$ wave is the same as an axon.
        \item Then the networks are equivalent.
        \item Also accept: $|D - g(x)| < \epsilon$ 
    \end{itemize}
    }
\end{figure}
\end{columns}
\end{frame}

\begin{frame}[fragile]{Proof sketch.}
\begin{figure}
    \centering
    \includegraphics[scale=0.3]{images/aan.png} 
    \caption{\textbf{Astrocyte Sudoku.}}
\end{figure}
\end{frame}


% ---------------------------------------------------------
\section[In practice.]{Practice.}
\begin{frame}[fragile]{Three kinds of astrocyte layers.}
\begin{figure}
\centering
\includegraphics[scale=0.3]{images/layers.png}     
\caption{I explored three fundamental kinds of astrocyte $Ca^{2+}$ waves [in PyTorch].}
\end{figure}
\end{frame}

\begin{frame}[fragile]{An astrocyte network.}
\begin{figure}
    \centering
    \includegraphics[scale=0.4]{images/glaidim.png} 
    \caption{
    \begin{itemize}
        \item Two equivalent neural (\textit{left}) and astrocyte networks (\textit{right}).
        \item For astrocyte computation \alert{width requires depth}.
    \end{itemize}}
\end{figure}
\end{frame}

\begin{frame}[fragile]{Methods.}
\begin{itemize}
    \item \alert{Task}: MINST digits
    \item \alert{Use}: \{\textit{slide}, \textit{spread}, \textit{gather}\} in pytorch
    \begin{enumerate}
        \item VAE, $N=(784,20)$
        \item Perceptron, $N=(20,30,10)$    
    \end{enumerate}
    \item Loss: Cross-entropy
    \item Optimizer: ADAM
\end{itemize}
\end{frame}

\begin{frame}[fragile]{Results.}
\begin{columns}
\column{0.4\textwidth}
    \centering
    \includegraphics[scale=0.2]{images/minst.png} 
\column{0.6\textwidth}
\begin{figure}
    \centering
    \includegraphics[scale=0.3]{images/results.png} 
    \caption{Test set performance (N=20 epochs)}
\end{figure}
\end{columns}
\end{frame}

\begin{frame}[fragile]{Results.}
\begin{figure}
    \centering
    \includegraphics[scale=0.25]{images/leak.png} 
    \caption{Effect of gliotransmitter diffusion on model performance}
\end{figure}
\end{frame}
% ---------------------------------------------------------
\section[Conclusions.]{Conclusions.}
\begin{frame}[fragile]{Conclusions.}
\begin{itemize}
\item Artifical astrocytes can compute any function, in theory.
\item An \alert{upper bound} for the performance of real astrocytes? 
\item Our simple model $Ca^{2+}$ waves can solve hard vision problems, in practice.
\end{itemize}
\end{frame}

\begin{frame}[fragile]{Future work.}
\begin{itemize}
\item Artificial upper bound $\rightarrow$ biological upper bound (\alert{help})
\item Recurrent waves (in collaboration)
\item Better motivated tasks (\alert{help})
\end{itemize}
\end{frame}

\begin{frame}[fragile]{Open science.}
\begin{itemize}
\item[Code] \url{github.com/CoAxLab/glia_playing_atari}
\item[Talk] \url{github.com/parenthetical-e/glia-talk-sfn-2019}
\item[] 
\item[] \alert{Thank you!}
\end{itemize}
\end{frame}


% ---------------------------------------------------------
% \begin{frame}[allowframebreaks]{References}
%   \bibliography{demo}
%   \bibliographystyle{abbrv}
% \end{frame}

% ---------------------------------------------------------
\end{document}
